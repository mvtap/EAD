\PassOptionsToPackage{unicode=true}{hyperref} % options for packages loaded elsewhere
\PassOptionsToPackage{hyphens}{url}
%
\documentclass[]{article}
\usepackage{lmodern}
\usepackage{amssymb,amsmath}
\usepackage{ifxetex,ifluatex}
\usepackage{fixltx2e} % provides \textsubscript
\ifnum 0\ifxetex 1\fi\ifluatex 1\fi=0 % if pdftex
  \usepackage[T1]{fontenc}
  \usepackage[utf8]{inputenc}
  \usepackage{textcomp} % provides euro and other symbols
\else % if luatex or xelatex
  \usepackage{unicode-math}
  \defaultfontfeatures{Ligatures=TeX,Scale=MatchLowercase}
\fi
% use upquote if available, for straight quotes in verbatim environments
\IfFileExists{upquote.sty}{\usepackage{upquote}}{}
% use microtype if available
\IfFileExists{microtype.sty}{%
\usepackage[]{microtype}
\UseMicrotypeSet[protrusion]{basicmath} % disable protrusion for tt fonts
}{}
\IfFileExists{parskip.sty}{%
\usepackage{parskip}
}{% else
\setlength{\parindent}{0pt}
\setlength{\parskip}{6pt plus 2pt minus 1pt}
}
\usepackage{hyperref}
\hypersetup{
            pdfborder={0 0 0},
            breaklinks=true}
\urlstyle{same}  % don't use monospace font for urls
\usepackage{longtable,booktabs}
% Fix footnotes in tables (requires footnote package)
\IfFileExists{footnote.sty}{\usepackage{footnote}\makesavenoteenv{longtable}}{}
\usepackage{graphicx,grffile}
\makeatletter
\def\maxwidth{\ifdim\Gin@nat@width>\linewidth\linewidth\else\Gin@nat@width\fi}
\def\maxheight{\ifdim\Gin@nat@height>\textheight\textheight\else\Gin@nat@height\fi}
\makeatother
% Scale images if necessary, so that they will not overflow the page
% margins by default, and it is still possible to overwrite the defaults
% using explicit options in \includegraphics[width, height, ...]{}
\setkeys{Gin}{width=\maxwidth,height=\maxheight,keepaspectratio}
\setlength{\emergencystretch}{3em}  % prevent overfull lines
\providecommand{\tightlist}{%
  \setlength{\itemsep}{0pt}\setlength{\parskip}{0pt}}
\setcounter{secnumdepth}{0}
% Redefines (sub)paragraphs to behave more like sections
\ifx\paragraph\undefined\else
\let\oldparagraph\paragraph
\renewcommand{\paragraph}[1]{\oldparagraph{#1}\mbox{}}
\fi
\ifx\subparagraph\undefined\else
\let\oldsubparagraph\subparagraph
\renewcommand{\subparagraph}[1]{\oldsubparagraph{#1}\mbox{}}
\fi

% set default figure placement to htbp
\makeatletter
\def\fps@figure{htbp}
\makeatother


\date{}

\begin{document}

\hypertarget{header-n2}{%
\section{Assignment 1}\label{header-n2}}

\hypertarget{header-n3}{%
\subsection{Introdução}\label{header-n3}}

No mundo actual existem cada vez mais dados disponíveis sobre o dia a
dia de um cidadão normal, quer seja através da utilização dos mais
diversos dispositivos como através da utilização de serviços colectivos
em grandes centros urbanos. De forma a tirar partido desta enorme
quantidade de dados para chegar a conclusões viáveis em tempo útil, é
necessário fazer uso dos métodos disponíveis para redução de variáveis e
explicação de variância. Com isto é possível aumentar o rendimento das
análises dos dados.\textless{}br/\textgreater{} Neste projecto foi
proposto a escolha de um \emph{dataset} com determinadas características
e a realização de diversos tipos de análises, univariada, bivariada e
multivariada, aos dados obtidos.\textless{}br/\textgreater{} Numa
primeira parte será exposto uma breve análise dos dados agrupados por
continentes, de forma a dar uma visão mais geral ao leitor, e também de
contextualizar com a realidade actual. De seguida, cada variável será
analisada com o intuito de se perceber como estão distribuídas.
Posteriormente, o alvo de estudo será a relação entre essas mesmas
variáveis e a sua influência, principalmente no \emph{Score} de cada
país.\textless{}br/\textgreater{} Na segunda e última parte serão
abordados os métodos de \emph{Factor Analysis} de forma a avaliar o peso
de cada variável nas restantes. Os dados serão normalizados de forma a
que os consigamos analisar na mesma escala de grandeza e avaliados sobre
a sua adequação. Para terminar, uma \emph{Principal Component Analysis}
será realizada e conclusões de como algumas variáveis são mais
relevantes do que outras.

\hypertarget{header-n5}{%
\subsection{Análise de dados}\label{header-n5}}

\hypertarget{header-n6}{%
\subsubsection{Introdução}\label{header-n6}}

Para a realização deste trabalho foi escolhido um dataset que traduz o
nível de felicidade, bem como outros indicadores, de diversos países,
relativo ao ano de 2019. Uma vez que cada entrada correspondia a um
país, decidiu-se agrupar os dados pelo continente ao qual os países
pertencem.\textless{}br/\textgreater{} Começando pelo \emph{Score}, este
é baseado nas respostas de um questionário sobre a avaliação da
qualidade de vida da população. Na questão, conhecida como \emph{Escada
de Cantril} é pedido que se imagine uma escada com 10 degraus (0 em
baixo e 10 no topo). O décimo degrau corresponde à melhor vida que o
questionado poderia ter, e o primeiro, a pior.

\begin{figure}
\centering
\includegraphics{/home/miguel/.config/Typora/typora-user-images/image-20210412194345893.png}
\caption{Média de Score por Continente}
\end{figure}

Na imagem acima podemos observar a média dos \emph{Scores} dos diversos
países pelo continente ao qual pertencem. Rapidamente se observa que
continentes onde se encontram os países mais desenvolvidos são os que,
em média, têm um \emph{Score} mais elevado. Na imagem abaixo podemos
observar a distribuição do \emph{Score} pelos diversos países do mundo.

\begin{figure}
\centering
\includegraphics{/home/miguel/.config/Typora/typora-user-images/image-20210412202442438.png}
\caption{Score dos diversos países}
\end{figure}

Existem outros factores que estão fortemente relacionados, mas sem
impacto, com o \emph{Score}, tais como o \emph{GDP per capita (PIB per
capita), Social support (apoios sociais), Healthy life expectancy
(esperança média de vida), Freedom to make life choices (liberdade),
Generosity (generosidade) e Perception of corruption (Corrupção)}. É
importante realçar que todos estes factores são também resultados de
questionários, expecto o PIB per capita e a esperança média de
vida.\textless{}br/\textgreater{} Uma nota para a variável
\emph{Corrupção}. Esta não representa o quão corrupto é um país, mas sim
a capacidade da população em detectar/identificar corrupção.

\begin{figure}
\centering
\includegraphics{/home/miguel/.config/Typora/typora-user-images/image-20210412195833285.png}
\caption{Média dos restantes factores por continente}
\end{figure}

Como esperado, todos os indicadores seguem o \emph{Score}. Relativamente
à generosidade e corrupção, a Oceania apresenta valores de média muito
mais elevados que os restantes continentes devido ao facto de a sua
amostra é constituída por apenas dois países de elevado índice de
desenvolvimento.\textless{}br/\textgreater{} Para concluir, através de
uma análise rápida dos dados agrupados por continente, fica retida a
ideia que os países considerados desenvolvidos vão obter resultados mais
elevados, e que esses mesmo países se encontram, na sua maioria, no
hemisfério norte.

\hypertarget{header-n14}{%
\subsubsection{Análise Univariada}\label{header-n14}}

Neste capítulo será realizada uma análise estatística a cada uma das
variáveis presentes no \emph{dataset}. Analisando a tabela abaixo
representada podemos retirar algumas conclusões:

\begin{itemize}
\item
  O número de países representados é de 156;
\item
  A média do \emph{Score} está próximo do meio da escala e o valor
  máximo é de 7.77 e mínimo de 2.85;
\item
  Metade dos países representados têm um \emph{Score} compreendido entre
  4.5 e 6.2;
\item
  Alguns países não têm dados relativos aos factores em causa, pois
  temos mínimos de 0;
\item
  As respectivas médias e medianas andam próximas, de onde concluímos
  que a distribuição dos valores será aproximadamente centrada.
\end{itemize}

\begin{longtable}[]{@{}llllllll@{}}
\toprule
& Score & GDP per capita & Social support & Healthy life expectancy &
Freedom to make life choices & Generosity & Perceptions of
corruption\tabularnewline
\midrule
\endhead
count & 156 & 156 & 156 & 156 & 156 & 156 & 156\tabularnewline
mean & 5.407 & 0.905 & 1.208 & 0.725 & 0.392 & 0.184 &
0.110\tabularnewline
std & 1.113 & 0.398 & 0.299 & 0.242 & 0.143 & 0.095 &
0.094\tabularnewline
min & 2.853 & 0.000 & 0.000 & 0.000 & 0.000 & 0.000 &
0.000\tabularnewline
25\% & 4.544 & 0.602 & 1.055 & 0.547 & 0.308 & 0.108 &
0.047\tabularnewline
50\% & 5.379 & 0.960 & 1.271 & 0.789 & 0.417 & 0.177 &
0.085\tabularnewline
75\% & 6.184 & 1.232 & 1.452 & 0.881 & 0.507 & 0.248 &
0.141\tabularnewline
max & 7.769 & 1.684 & 1.624 & 1.141 & 0.631 & 0.566 &
0.453\tabularnewline
IQR & 1.640 & 0.629 & 0.396 & 0.334 & 0.199 & 0.139 &
0.094\tabularnewline
skew & 0.011 & -0.385 & -1.134 & -0.613 & -0.685 & 0.745 &
1.650\tabularnewline
mad & 0.916 & 0.332 & 0.236 & 0.199 & 0.116 & 0.075 &
0.069\tabularnewline
kurt & -0.608 & -0.769 & 1.229 & -0.302 & -0.068 & 1.173 &
2.416\tabularnewline
\bottomrule
\end{longtable}

Analisando o valor de \emph{skewness} concluímos que não temos nenhuma
das variáveis totalmente simétricas quanto à sua distribuição. Por
outras palavras, a \emph{skewness} traduz a falta de simetria das
distribuições. Adicionando a análise de \emph{Kurtosis}, podemos
identificar alguns casos em que a presença de \emph{outliers} é bastante
provável (\emph{p.e. Perceptions of corruption}).

\includegraphics{/home/miguel/.config/Typora/typora-user-images/image-20210412232710214.png}

Recorrendo à análise gráfica as conclusões convergem. Todas as
distribuições são aproximadamente simétricas, com a presença dos
expectáveis \emph{outliers}. Alguns deles correspondem aos valores em
falta de alguns países. Contudo, a sua existência é única, e por isso,
não significativa.\textless{}br/\textgreater{} Como esperado, na
corrupção temos a presença de um número maior de \emph{outliers}, o que
vai de encontro ao respectivo valor de
\emph{Kurtosis}.\textless{}br/\textgreater{} Concluindo, não existe uma
grande disparidade em relação ao \emph{Score} e ao \emph{GDP per capita
(PIB per capita)} dentro dos países analisados. Relativamente ao
\emph{Social support (apoios sociais)} e à \emph{Healthy life expectancy
(esperança média de vida)}, as populações têm uma visão positiva do seu
país. O mesmo pode ser dito da \emph{Freedom to make life choices
(liberdade)}. Por último, \emph{Generosity (generosidade)} e
\emph{Perception of corruption (Corrupção)}, com uma performance menos
boa. A corrupção, como foi dito anteriormente, traduz a percepção de
corrupção. Logo podemos concluir que a presença de \emph{outliers}
corresponde às populações dos países mais desenvolvidos, possivelmente
devido ao maior nível educacional das mesmas.

\hypertarget{header-n148}{%
\subsubsection{Análise Bivariada}\label{header-n148}}

Neste capítulo será analisada a relação entre as variáveis. Os casos
mais pertinentes serão quase sempre relativos à relação com o
\emph{Score}. No entanto, a análise de outras relações pode ser útil de
forma a clarificar, indirectamente, alguns tópicos.

\begin{figure}
\centering
\includegraphics{/home/miguel/.config/Typora/typora-user-images/image-20210412235224302.png}
\caption{}
\end{figure}

Analisando a linha do \emph{Score} vemos que existe uma correlação
positiva com quase todas as variáveis. Contudo, a generosidade
practimente não têm influência, e a corrupção apenas se consegue
detectar a presença de correlação em valores
elevados.\textless{}br/\textgreater{} Algumas correlações positivas
esperadas comprovam-se, tal como \emph{PIB per capita} e \emph{Apoio
social}, \emph{PIB per capita} e \emph{esperança média de vida}, e por
último \emph{Apoio social} e \emph{esperança média de vida}. Nota para o
facto de não se conseguir identificar nenhuma correlação
negativa.\textless{}br\textgreater{} Antes de passar para a análise do
mapa de correlações, notar que existe uma visível correlação positiva
entre a \emph{percepção de corrupção} e \emph{liberdade}, o que de certa
forma é expectável, tendo em conta que os países no mundo com maior
liberdade são também os que apresentam maior transparência nos seus
processos governamentais.

\includegraphics{/home/miguel/.config/Typora/typora-user-images/image-20210413001206315.png}

Analisando o mapa de correlações vemos que vai de encontro às conclusões
obtidas anteriormente. Fortes correlações positivas entre as quatro
primeiras variáveis, a muito reduzida presença de correlações negativas
( e com valores muito próximos de zero) e a correlação positiva, embora
não relevante, entre a \emph{percepção de corrupção} e \emph{liberdade}.

\hypertarget{header-n154}{%
\subsection{Factor Analysis}\label{header-n154}}

\hypertarget{header-n155}{%
\subsubsection{Normalização}\label{header-n155}}

Uma vez que as variáveis têm diferentes escalas, por exemplo o
\emph{Score} varia entre 0 e 10, ao passo que a perceção de corrupção
encontra-se entre 0 e 1 (uma vez que se trata de uma média que avalia as
resposta a uma pergunta com a possibilidade de responder sim (1) ou não
(0)) procedeu-se à normalização dos valores (com remoção da média e
variância unitária) em todas as variáveis com exceção da \emph{Country
or Region}, visto se tratar de uma variável categórica nominal.

\hypertarget{header-n157}{%
\subsubsection{Testes de adequação}\label{header-n157}}

Inicialmente foi aplicado o teste de \emph{Bartlett} sobre os dados
normalizados, sendo a hipótese nula a matriz de correlações trata-se de
uma matriz de identidade, o que indicaria que as variáveis seriam não
correlacionadas e portanto, não adequadas para \emph{factor analysis}
(FA) \footnote{IBM, «IBM Docs», Out. 24, 2014.
  www.ibm.com/docs/en/spss-statistics/23.0.0 (acedido Abr. 13, 2021)}.
Obteve-se um valor de chi-quadrado de aproximadamente 656 e o nível de
significância obtido foi de 0.0 (\(5*10^{-126}\)) indicando a rejeição
da hipótese e, portanto, que os dados são adequados ao tipo de análise
em questão.\textless{}br/\textgreater{} Foi, também, realizado o teste
de adequação \emph{Kaiser-Meyer-Olkin's} (KMO), obtendo-se o valor de
aproximadamente 0.84 o que evidencia uma forte adequação à realização de
FA. A tabela seguinte exibe os valores de \emph{Measure of Sampling
Adequacy} (MSA) para as variáveis em análise. Todos as variáveis possuem
um MSA superior a 0.5, tendo sido portanto mantidas para análise
\footnote{IBM, «Kaiser-Meyer-Olkin measure for identity correlation
  matrix», Abr. 16, 2020.
  https://www.ibm.com/support/pages/kaiser-meyer-olkin-measure-identity-correlation-matrix
  (acedido Abr. 13, 2021).}.

\begin{longtable}[]{@{}lllllll@{}}
\toprule
\emph{Score} & \emph{GDP} & \emph{Social Support} & \emph{Healthy life
exp.} & \emph{Freedom} & \emph{Generosity} &
\emph{Corruption}\tabularnewline
\midrule
\endhead
0.855 & 0.827 & 0.871 & 0.862 & 0.829 & 0.596 & 0.752\tabularnewline
\bottomrule
\end{longtable}

\hypertarget{header-n177}{%
\subsubsection{PCA e análise}\label{header-n177}}

Procedeu-se à aplicação da PCA sobre os dados normalizados, tendo-se
obtidos os \emph{eigenvalues} apresentados na tabela seguinte:

\begin{longtable}[]{@{}llll@{}}
\toprule
& \emph{Eigenvalue} & Fracção de Variância Explicada (\%) & Fracção
acumulada (\%)\tabularnewline
\midrule
\endhead
1 & 3.837141 & 54.46 & 54.46\tabularnewline
2 & 1.436346 & 20.39 & 74.85\tabularnewline
3 & 0.616839 & 8.76 & 83.61\tabularnewline
4 & 0.559896 & 7.95 & 91.56\tabularnewline
5 & 0.263794 & 3.74 & 95.30\tabularnewline
6 & 0.173418 & 2.46 & 97.76\tabularnewline
7 & 0.157726 & 2.24 & 100.00\tabularnewline
\bottomrule
\end{longtable}

Na imagem ilustrada podemos observar a representação gráfica da anterior
tabela:

\begin{figure}
\centering
\includegraphics{/home/miguel/.config/Typora/typora-user-images/image-20210413230831123.png}
\caption{}
\end{figure}

Verifica-se que a primeira e segundas componentes explicam
aproximadamente 54\% e 20 \% da variância, totalizando 74.85\% da
variância total, sendo que para as componentes seguintes existe uma
queda brusca relativamente a estas. Aplicando o critério de Kaiser
(selecionar apenas as componentes a que corresponde um \emph{eigenvalue}
superior a 1) foram selecionadas as primeiras duas componentes (PC1 e
PC2).\textless{}br/\textgreater{} Na tabela seguinte apresentam-se os
\emph{loadings} das primeiras 2 componentes, que representam as
correlações entre as componentes e as variáveis, valores (absolutos)
mais elevados de correlação encontram-se realçados. Estes valores
refletem a importância de cada variável nas componentes.

\begin{longtable}[]{@{}lll@{}}
\toprule
Features & PC1 & PC2\tabularnewline
\midrule
\endhead
\emph{Score} & \texttt{-0.475861} & -0.028371\tabularnewline
\emph{GDP} & \texttt{-0.454825} & -0.213377\tabularnewline
\emph{Healthy life exp} & \texttt{-0.436582} & -0.207148\tabularnewline
\emph{Social support} & \texttt{-0.450150} & -0.177856\tabularnewline
\emph{Freedom} & -0.332201 & 0.362130\tabularnewline
\emph{Generosity} & -0.048232 & \texttt{0.693809}\tabularnewline
\emph{Corruption} & -0.246511 & \texttt{0.516346}\tabularnewline
\bottomrule
\end{longtable}

Seguidamente podemos observar os dados dos \emph{loadings} no círculo de
correlação. Como se pode visualizar a PC1 está mais correlacionada com
as variáveis \emph{Score, GDP, Social support and Healthy life exp.}, ao
passo que a PC2 está mais ligada às 3 restantes
variáveis.\textless{}br/\textgreater{} Foi aplicada FA com duas
componentes tendo sido utilizada uma rotação (de forma a melhorar a
interpretação) do tipo \emph{varimax} e o método de eixo principal para
realizar a extração, tendo-se obtido as \emph{loadings, communalities} e
variância específica representadas na seguinte tabela:

\begin{longtable}[]{@{}lllll@{}}
\toprule
& Factor 1 & Factor 2 & \emph{Communalities} & Variância
específica\tabularnewline
\midrule
\endhead
\emph{Score} & 0.886962 & 0.278875 & 0.864474 & 0.14\tabularnewline
\emph{GDP} & 0.922187 & 0.056855 & 0.853661 & 0.15\tabularnewline
\emph{Healthy life exp.} & 0.886130 & 0.051952 & 0.787925 &
0.21\tabularnewline
\emph{Social support} & 0.899390 & 0.093791 & 0.817701 &
0.18\tabularnewline
\emph{Freedom} & 0.466562 & 0.624670 & 0.607894 & 0.39\tabularnewline
\emph{Generosity} & -0.188509 & 0.812598 & 0.695852 &
0.30\tabularnewline
\emph{Corruption} & 0.247258 & 0.742319 & 0.612174 & 0.39\tabularnewline
\bottomrule
\end{longtable}

Pela análise da tabela podemos concluir que o Factor 1 está fortemente
relacionado com as primeiras 4 variáveis ao passo que o Factor 2
encontra-se mais correlacionado com as restantes variáveis. Em termos de
\emph{communalities} verifica-se que uma fração significativa das
variáveis é explicada pelo factores presentes. Na seguinte está
representado o círculo de correlação das \emph{loadings} obtidas.

\begin{figure}
\centering
\includegraphics{/home/miguel/.config/Typora/typora-user-images/image-20210413235253400.png}
\caption{}
\end{figure}

Na próxima figura encontra-se representado o gráfico dos indivíduos
(países) quando a eles é aplicada a transformação imposta pelo modelo
determinado pela FA. Verifica-se que, por exemplo, os países europeus
têm a tendência de estar mais na positiva do eixo das abcissas,
indicando que estes (provavelmente) têm um maior \emph{Score}, esperança
média de vida saudável, GDP e suporte social (uma vez que a correlação
destas variáveis é mais forte com a componente 1. No caso dos países
africanos verifica-se que maioritariamente ocupam as posições do gráfico
mais à esquerda (indicando maus parâmetros em termos de \emph{score},
esperança média de vida saudável, GDP e suporte social).

\begin{figure}
\centering
\includegraphics{/home/miguel/.config/Typora/typora-user-images/image-20210413235431691.png}
\caption{}
\end{figure}

\hypertarget{header-n310}{%
\subsection{Conclusão}\label{header-n310}}

Com a realização deste trabalho é possível retirar a conclusão de que os
métodos estatísticos abordados são de uma enorme importância para a
escolha de variáveis. Embora o \emph{dataset} escolhido não fosse muito
extenso, foi possível analisar a importância de cada uma das variáveis e
o peso das mesmas.\textless{}br/\textgreater{} Encontraram-se algumas
dificuldades, nomeadamente na procura de suporte para alguns métodos na
linguagem \emph{Python}, pois ainda não é tão abrangente neste capítulo
como o \emph{R}.\textless{}br/\textgreater{} Relativamente aos dados,
podemos concluir que existem variáveis mais importantes para que a
resposta da população sobre o índice de felicidade seja mais elevada,
tais como o \emph{GDP} ou o \emph{Social support}. Ambas seriam
provavelmente dedutíveis sem esta análise. Contudo, outras variáveis que
geralmente consideraríamos relevantes, concluiu-se que não o são.

\hypertarget{header-n312}{%
\subsection{Referências}\label{header-n312}}

\end{document}
